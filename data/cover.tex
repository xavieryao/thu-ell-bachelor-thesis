\thusetup{
  %******************************
  % 注意:
  %   1. 配置里面不要出现空行
  %   2. 不需要的配置信息可以删除
  %******************************
  %
  %=====
  % 秘级
  %=====
  %
  %=========
  % 中文信息
  %=========
  ctitle={中国计算机科学研究生学术论文写作中引用行为的研究},
  cdegree={文学学士},
  cdepartment={外国语言文学系},
  cmajor={英语语言文学},
  cauthor={姚沛然},
  csupervisor={颜奕\hskip0.6cm 副教授},
  % 日期自动使用当前时间,若需指定按如下方式修改:
  % cdate={超新星纪元},
  %
  %=========
  % 英文信息
  %=========
  etitle={Citation Behaviors of Chinese Computer Science Graduate Students in Research Paper Writing},
  % 这块比较复杂,需要分情况讨论:
  % 1. 学术型硕士
  %    edegree:必须为Master of Arts或Master of Science(注意大小写)
  %             “哲学、文学、历史学、法学、教育学、艺术学门类,公共管理学科
  %              填写Master of Arts,其它填写Master of Science”
  %    emajor:“获得一级学科授权的学科填写一级学科名称,其它填写二级学科名称”
  % 2. 专业型硕士
  %    edegree:“填写专业学位英文名称全称”
  %    emajor:“工程硕士填写工程领域,其它专业学位不填写此项”
  % 3. 学术型博士
  %    edegree:Doctor of Philosophy(注意大小写)
  %    emajor:“获得一级学科授权的学科填写一级学科名称,其它填写二级学科名称”
  % 4. 专业型博士
  %    edegree:“填写专业学位英文名称全称”
  %    emajor:不填写此项
  % 日期自动生成,若需指定按如下方式修改:
  % edate={December, 2005},
  %
  % 关键词用“英文逗号”分割
  ckeywords={引用行为, 外语写作, 学术英语教学},
  ekeywords={citation behaviros, L2 writing, EAP pedagogy}
}

% 定义中英文摘要和关键字
\begin{cabstract}
  恰当引用文献是研究生必备的基本学术素养之一。然而,引用文献涉及到多种认知过程、多方面语言知识,因此是一个较难培养的素养。近期的研究表明,通过文本分析研究引用行为可帮助学术英语教师指导学生引用文献、帮助培养学生引用文献这一学术素养;另外,学术英语教师对学生引用文献的指导应因学科而异。因此,为帮助学术英语教师指导中国计算机科学专业研究生恰当引用文献,本研究分析了11个中国的一年级计算机科学专业研究生的研究论文,并以此研究他们的引用行为。研究使用的语料库由10篇研究论文构成,这些论文是所研究的研究生在旨在指导学生写作和发表学术论文的一年级学术英语写作课程的作业。本研究从引用的形式、目的、质量三个角度对语料库进行了分析,并总结了被研究群体引用行为的特点和不足。基于引用行为分析的结果,本研究还讨论了对教师指导学生学术英语写作的启示及未来研究的可能方向。
\end{cabstract}

% 如果习惯关键字跟在摘要文字后面,可以用直接命令来设置,如下:
% \ckeywords{\TeX, \LaTeX, CJK, 模板, 论文}

\begin{eabstract}
  Using citations is one of the basic but vital academic literacies for graduate students to succeed in academic communities. However, this literacy is not easy to develop as it involves multiple cognitive and linguistic processes. Recent studies have shown that text analysis of citation behaviors could help EAP teachers instruct students on using citations, and the instructions should be adapted according to the group of writers. Therefore, we conducted an analysis of the citation behaviors of a group of 11 first-year Chinese computer science graduate students based on their research papers, in order to help teachers instruct this group of writers. The citation quality, forms and functions were analyzed using a corpus composed of 10 research papers written by these students as part of the requirements of a course aimed at assisting first-year graduate students writing and publishing research papers. The analysis has revealed characteristics and deficiencies of the citation use of this group of students.  In light of the findings of this study, implications for the instruction of academic English writing are discussed, as are considerations for future research.
\end{eabstract}

% \ekeywords{\TeX, \LaTeX, CJK, template, thesis}
