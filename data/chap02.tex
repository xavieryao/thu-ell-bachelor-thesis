\engchapter{Review of the Literature}
\label{chap:lit_review}

\section{Citation Behavior Analysis}
The theory of intertextuality suggests that an academic text is composed by organizing other texts in order to generate new knowledge claims \citep{fairclough_intertextuality_1992, shi_textual_2010}. Citing others in academic writing requires the ability to critically integrate the ideas in primary sources with the writer’s own stance \citep{wette_source_2017}. Various aspects of citation behaviors have been studied, including forms, rhetorical functions, quality and stances.

\citet{swales_citation_1986} leads a pioneering role in research on citation analysis. Swales proposed a simple typology of citation form analysis that categorized citation forms as integral and non-integral citations. Integral citations consist of a part of a sentence, while non-integral ones are usually in brackets outside a sentence. However, in this early work, only the number of positive citations was investigated, which is clearly not enough. \citet{thompson_looking_2001} further conducted a pivotal and comprehensive study on the forms of citations in PhD dissertations. He extended Swale’s typology by subcategorization of citation functions. Thompson also investigated reporting verbs and their tenses and voices, which were also studied by \citet{manan_analysis_2014}. However, these studies only focused on the linguistic realization of citations.

\citet{petric_rhetorical_2007}́ has broadened the range of citation behavior analysis to rhetorical functions. Although Thompson’s framework has already included functions in his sub- categorization of non-integral citations, Petrić’s typology has a finer granularity, with eight functions in total. \citet{samraj_form_2013} pointed out that Petrić’s analysis framework is more reasonable than Thompson’s, as surface forms and rhetorical functions should be independent with each other.

The subjects of citation behavior analysis have included nearly all academic genres. For novice writers, first-year \citep{lee_citation_2018} and second-year \citep{wette_source_2017} undergraduates’ disciplinary assignments were studied. More professional academic writing, including grant proposals \citep{fazel_citation_2015}, master theses and doctoral dissertations. Most of the studies were based on textual analysis. Apart from texts, some of them \citep{fazel_citation_2015,harwood_interview-based_2009,wette_source_2017} included interviews with the writers to better identify the citation functions, as textual analysis itself may not reveal the true intention of the author \citep{harwood_interview-based_2009}.

To study the distinctions between different groups of writers, there are also comparative studies. Some of them compared high-quality academic writing with low- quality ones. For example, \citet{petric_rhetorical_2007} compared high- and low-rate master theses to see their differences in rhetorical functions of citations. Similarly, \citet{samraj_form_2013} compared master’s theses and research articles and set research articles as a baseline of “good” writing. When it comes to L2 writers, \citet{mansourizadeh_citation_2011} compared non-native experts and novice native writers in science disciplines. In addition to groups with different writing proficiencies, \citet{hyland_academic_1999} compared research articles across different disciplines and concluded that there were significant differences across disciplines. \citet{hu_disciplinary_2014} studied the ethnolinguistic influences on citation use, and concluded that within the same discipline, citation behaviors might be different among different ethnolinguistic groups.

\section{Challenges of Citation Use for EAP Students}
Citing is not as simple as “providing a name and a data”, but it requires a decision process to extract the meaning out of a source \citep[p.~21]{shi_textual_2010}. This ability to comprehend the source texts, integrate them with the writer’s stance and express it is challenging for nearly all novice writers. For EAP students, making proper citations is inherently challenging. First of all, making proper citations is cognitively challenging. It might be easy to paraphrase a single source, but writers also have to identify and synthesize the connections between multiple sources through the cognitive processes of reflection, planning, evaluation and revision \citep{bereiter_psychology_2013,hyland_drawing_2009,kirkland_maximizing_1991,segev-miller_cognitive_2007}. Linguistically, the knowledge of English syntax and lexicon must be involved \citep{mayes_quotation_1990}.

The proficiency of using citations is also limited by discipline knowledge. Without a relatively thorough understanding of domain knowledge, writers not only have trouble in selecting appropriate sources, but also hold a relatively abased stance in referring to others. In terms of rhetorical functions of citations, novice writers use citations mainly to supply content rather than to support the writer’s own ideas \citep{mansourizadeh_citation_2011,petric_rhetorical_2007}. Rather than interpreting the content of sources, inexperienced writers would use citations just for display \citep{bereiter_psychology_2013}. Without an in-depth understanding of the domain knowledge, inexperienced writers face difficulties in making connections with the large network of sources \citep{hyland_representing_2005}.

Another set of literature focused on unintended plagiarism of novice writers. \citet{wette_source_2017} summarized that being unskilled in writing, novice writers unintentionally commit plagiarism through the form of patchwriting. Patchwriting means to copy some short word strings, or to copy a whole sentence from the original text and replace some words with synonyms \citep{li_two_2012, pecorari_good_2003}. This is one of the major issues toward making high-quality citations for non-expert writers \citep{wette_source_2017}.

The challenges that L2 writers might face in using citations have also been extensively studied. In a more general term, L2 writers with lower IELTS scores rely more on writing supports compared with native writers after they got admitted into English-medium universities \citep{ridley_tracking_2006}. Specifically with citations, it has been identified that L2 writers demonstrate a smaller vocabulary of reporting verbs \citep{schmitt_writing_2007, thompson_evaluation_1991} which definitely will affect the fluency and eloquence of their academic writing. \citet{shi_chinese_2018} reviewed the various difficulties Chinese academic writers might face, and concluded some findings from previous research: Chinese students tend to borrow many words from the sources \citep{shi_textual_2004}, and when comparing Chinese L2 writers in an MBA program with their L1 peers, \citet{yang_exploring_2003} found that Chinese writers are influenced by their L1 (Chinese) when summarizing in terms of thinking process as well as writing skills. In \citeauthor{hyland_authority_2002}’s \citeyearpar{hyland_authority_2002} comparison study of Hong Kong undergraduates and L1 scholars, he found Hong Kong undergraduates were less likely to express ownership of their own views, which Hyland thought as a reflection of Chinese culture on writing. \citet{liu_attitude_2009} compared academic writings in Chinese and English and found fewer affect and judgment items in Chinese academic texts. \citet{shi_chinese_2018} also made a summary of related research that Chinese scholars use fewer citations but use more integral citations where the emphasis is on the original writer.