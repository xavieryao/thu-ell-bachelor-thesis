\engchapter{Methods}
\label{chap:methods}

\section{Overview}
This small-scale qualitative study of research papers written by Chinese computer science graduate students uses textual analysis to analyze the key characteristics of the citation behaviors in these papers from the perspectives of citation quality, forms, and functions. The whole study will center on this research question: what are the characteristics of the citation behaviors of Chinese CS graduate students, regarding citation quality, citation forms, and citation functions? Based on the findings, we will also discuss the deficiencies of Chinese CS graduate students, the differences in citation behaviors between Chinese CS graduate students and the groups studied in previous research, what leads to these deficiencies and differences, and how can EAP teachers respond to these deficiencies and differences.

\section{Text Collection}
In total, n = 11 texts were collected, and all the research papers are in the discipline of computer science. All the texts collected in this study are from a top-tier research university in China, where Chinese is the instruction language. The writers of these research papers are first-year doctoral students attending the English for academic purposes course for doctoral students, and these texts served as an assignment of that course. These research papers were provided directly by the teachers of the EAP writing courses. By the time of this research, 6 out of 11 of the papers have already been published as peer-reviewed papers. All of writers use Chinese as their first language. These writers have received some training in writing research papers in English, but are not as proficient as senior doctoral students and graduated researchers.

\section{Text Analysis}
\subsection{General Principles}
The citations in all the texts were identified using the standards proposed by \citet{hyland_academic_1999}. That is, citations were first searched by the conventional signals, for example, author’s name and publish year in parenthesis, or a number in squared brackets. Then, all the names in the bibliography, third-person pronouns, and generalized terms of agents were examined for the citations without explicit marks.

The tags of those citations were obtained through inductive discourse analysis \citep{thomas_general_2006}, as is used in \citeauthor{samraj_form_2013}'s \citeyearpar{samraj_form_2013} recent study. By using the scheme, the texts were read and coded repeatedly in order to obtain better tagging standards with a clearer distinction between different tags. The texts were finally analyzed with the finalized standard.

\subsection{Coding Scheme for Citation Quality}
We assessed citation quality in four aspects: whether the citation should be considered as plagiarism, whether the writer can contextualize different sources, the citation density, and the sources used by the writers. During the coding process, we marked references to other works without citations, which signal plagiarism. The source of a citation was recorded, including conference paper, journal paper, preprint paper, book chapter, and other sources. We also calculated the number of citations per 1000 words as the citation density. Finally, every citation was also categorized as referring to only one single source, or multiple sources.

We measured the quality of sources using the H5-index of the venue and the times cited of the article. H-index is the maximum value of number \textit{h} such that at least \textit{h} papers within a group of papers were each cited \textit{h} or more than \textit{h} times \citep{hirsch_index_2005}, and it can be used as a unified metric to measure the impact of a journal or a conference \citep{braun_hirsch-type_2006}. H5-index is a variant of H-index that only uses data in the recent 5 years. The H5-indices of the sources were retrieved from the CS conference rankings of AMiner \citep{tang_arnetminer:_2008}, which includes CS conferences and journals with an H5-index greater than 13. Sources that are not listed in AMiner CS conference rankings but are journals or conferences proceedings published by ACM (the Association for Computing Machinery) or IEEE (the Institute of Electrical and Electronic Engineers) were coded as \textit{“I/A”}. Sources that are PhD theses, academic monographs, technical reports, or international standards were coded as \textit{“other reliable”}. Other sources were coded as \textit{“others”}. For each source cited, we retrieved how many times it was cited as of May 14, 2019 from Google Scholar.

\subsection{Coding Scheme for Citation Forms}

Following \citeauthor{thompson_looking_2001}'s \citeyearpar{thompson_looking_2001} framework, the surface form of a citation can be tagged as integral (part of a sentence) and non-integral (annotated after a sentence). Non-integral citations were further categorized as verb-controlling and naming. Verb-controlling citations control a verb, for example, \textit{“Thompson (2001) investigated the citation behaviors of PhD theses”}. The other form, naming, works as a modifier (\textit{“The work of Thompson (2001)”}) or a free-standing noun phrase followed by a linking verb (\textit{“The work of Thompson (2001) is the pivotal work of citation analysis”}).

In addition to the citation types regarding the surface forms, the reporting verbs and expressions, and position of placement, i.e. in which part of the research paper (introduction, literature review, method, discussion, etc.) the citation appears, were also recorded.

\subsection{Coding Scheme for Citation Functions}
This study follows the typology of rhetorical functions of \citet{petric_rhetorical_2007} to code citation functions, since that scheme was adopted in most recent works. There are in total nine categories of citation functions: attribution, exemplification, further reference, statement of use, application, evaluation, establishing links between sources, comparison or interpretation with other sources, and others. Citations of type attribution are to attribute information or activities to a source. Exemplification is used to provide examples. “Further references” often follows \textit{“see”} to provide detailed information in other sources. Citations that indicate that some works are used to certain purposes are classified as statement of use. Citations of type “application” use propositions in other works for the writer’s own purposes. If other works are evaluated using evaluative language, the citation function is evaluation. To present different views on a topic, writers use citations of type “establishing links between sources”.