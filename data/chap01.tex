\engchapter{Introduction}
\label{chap:intro}

\section{Background}
Using citations is one of the basic but essential literacies in academic writing. It is a fundamental tool to interact with the academic community. Most writers in the academia need proper references to other sources to introduce the background of the problem, provide support for the writer’s argument, compare the results with previous works, or to achieve other rhetorical functions \citep{petric_rhetorical_2007}.

However, novice writers, regardless of their deficiencies in academic writing, face challenges in properly citing others \citep{pecorari_plagiarism_2014}. Graduate students pursuing a research degree aim to enter the academic community. To communicate with and get involved in academic communities, they have to master proper citing practices to “contextualize their own arguments” \citep[p. 48]{wette_source_2017}. Recently, there have been several studies that rely on textual analyses to identify the core characteristics of citation behaviors in academic writing. The genres under study include course assignments, research papers, master theses, and doctoral dissertations. Researchers also conducted comparative studies to compare the citation behaviors between novice writers and professional writers, and between first language (L1) writers and second language (L2) writers. \citet{duff_second_2007} and \citet{lave_situated_1991} have shown that it would be very helpful for graduate students to learn about skilled citing practices in order to participate in their own academic community. Therefore, the findings of these studies can be used to guide the development of pedagogy for English for academic purposes (EAP), in order to further help graduate students involve in academic communities.

Additionally, textual analysis of the citation behaviors of L2 writers can help identify features of novice writers of a specific group. As \citet{thompson_looking_2001} suggested, citation behaviors differ from discipline to discipline. And it was further suggested by \citet{hu_disciplinary_2014} that within the same discipline, citation behaviors may vary subject to different ethnolinguistic groups. Therefore, it is necessary to conduct textual analyses specific to certain groups of writers. Most graduate programs in China use Chinese as the instruction language, therefore it is much more challenging to enter communities as English is the commonly used language in academia. To the best of my knowledge, although there has been an increasing interest in citation behavior analysis, no research has been published yet to study the citation behaviors of Chinese graduate students in computer science. Most similar studies were conducted in institutions where English is used as the instruction language, and these studies were not focused on an engineering discipline such as computer science. It would be intriguing to investigate the citation behaviors in academic writing in an environment where English is not used as the instruction language.

To gain the knowledge of citation behaviors of the group of Chinese graduate students in computer science, this study aims to analyze the citations in 11 research papers written by first-year graduate students in computer science and related disciplines through text analysis. We chose Chinese graduate students in computer science for the following reasons. First of all, as stated before, citation behaviors may vary accroding to disciplines or ethnolinguistic groups. Therefore it is necessary to conduct citation analyses on each different group. And to the best of our knowledge, there is currently no published research on the citation behaviors of Chinese CS graudate students regarding the aspects we would like to discuss in this study. Secondly, analyses on Chinese CS graduate students could possibly provide insights for the features of L2 (especially Chinese) writers, novice EAP writers, and science and engineering students.

\section{Aims}
This study aims at investigating the citation behaviors of second language writers when writing research papers from three aspects, namely, citation forms, citation functions and citation quality, and to compare these behaviors with previous reports on “successful” academic writings (research papers and degree theses). Expectedly, the findings of this study would reveal deficiencies and shortcomings of Chinese L2 writers in writing academic papers, so as to facilitate the teaching of EAP for L2 writers in relevant engineering disciplines.

\section{Structure of the Thesis}
This thesis is organized in the following way. Chapter \ref{chap:intro} gives an introduction to the study of this thesis. Chapter \ref{chap:lit_review} reviews works related to this study, including those on the characteristics and challenges of citations in academic writing, citation behaviors of academic groups, and various methodologies to analyze citations. Chapter \ref{chap:methods} describes the dataset used in this study and the coding schemes for citation form, function and quality analysis. Chapter \ref{chap:results} describes the results of the analyses as well as some possible explanations of some phenomena found. In Chapter \ref{chap:conclusion}, we conclude the whole thesis by summarizing the major findings of this study, pointing out the limitations and suggestions for future studies, and providing some implications for EAP teachers.